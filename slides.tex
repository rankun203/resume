\documentclass{seminar}
% \newif\ifpdf\ifx\pdfoutput\undefined\pdffalse\else\pdfoutput=1\pdftrue\fi
\usepackage{slidesec,fancybox}
% \usepackage{RuG2e}
\usepackage[pdftex,colorlinks]{hyperref}%pdftex driver for hyperref
                                        %colorlinks enables colored links
\usepackage{acronym}
\slideframe{Oval}

\begin{document}

% Define acronyms
\acrodef{LLM}{Large Language Model}

\begin{slide}
% \textcolor{blue}{\wapen}
\textcolor{red}{hi, colors are also available. This only works if the
optional argument colorlinks of hyperref is invoked}
% \textcolor{blue}{\balk}
\end{slide}

\begin{slide}
\textbf{Information Retrieval}

User query quality impacts retrieval effectiveness. This study categorizes
thousands of user queries spanning one hundred topics into three groups -- low,
medium, and high quality -- based on NDCG@10 scores.
% \ma{question/problem}
The study investigates the impact of fusing search results of
\ac{LLM}-generated query variants with results retrieved from user queries
drawn from the three groups, similar to a collaborative search approach where
users with diverse queries collaborate in locating relevant information. 
% \ma{results}
The findings indicate that a traditional search system can be significantly
improved by fusing results for low-quality queries, 
% \ma{the impact of results}
offering a promising solution for users who struggle to find relevant
information, particularly in contexts where advanced search systems are
impractical due to technical or resource constraints, or where access to query
logs are unavailable.
\end{slide}
\end{document}
